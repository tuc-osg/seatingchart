\documentclass[
%load=tucseating,
babel=ngerman
]{skdoc}
\usepackage[room=A10.201]{tucseating}
\microtypesetup{disable=true}

\package{\packagename}
\version{\packageversion}
\repository{https://github.com/tuc-osg/tucseating}
\author{Matthias Werner}
\email{matthias.werner@informatik.tu-chemnitz.de}
\title{Das \thepkg-Paket}


\begin{document}
\maketitle

\begin{abstract}
    tucseating ermöglicht es, einfach Sitzpläne zu
generieren, wie sie z. B. für Prüfungen benö-
tigt werden. Eine Reihe verschiedener automa-
tischer Platzierungsschemata sind vordefiniert,
aber man kann auch feingranular eigene Plat-
zierungen vornehmen.
Während das Paket zunächst für den internen
Gebrauch an der TU Chemnitz gedacht ist und
(einige) vordefinierte Raumpläne enthält, sind
einerseits sowohl die Raumdaten leicht erweiter-
oder ersetzbar, andererseits können Räume auch
ad hoc erstellt werden.
\end{abstract}
\tableofcontents

\section{Einführung}
Für die Durchführung von Prüfungen benötigen wir mitunter Sitzpläne.
So haben sich über die einige mit Pläne  in Form von TikZ-unterstützten
\LaTeX-Dateien angesammelt. 
Je nach Anzahl der Studierenden in einer Prüfung (und für wie groß wir die
Gefahr eines Betrugsversuches bewerten) nutzen wir unterschiedliche
Platzierungsschemata, so dass wir die Dateien in anpassen müssen.
Außerdem wird uns von Zeit zu Zeit ein neuer Raum zugewiesen, für den wir noch
keine Pläne haben.

Das \thepkg-Paket 
\begin{itemize}
  \item ermöglicht eine schnelle unde einfach Erstellung von Sitzpläne erstelltn;
  \item trennt das Raumlayout und das Platzierungsschema voneinander;
  \item bietet eine Reihe von Standardschemata für die Platzierung an;
  \item enthält bereits eine Anzahl vordefinierter Raumlayouts;
  \item erlaubt eine \emph{Ad-hoc}-Erstellung neuer Räume und Sitzschemata.
\end{itemize}

\section{Abhängigkeiten}
Das \thepkg-Paket arbeitet nur mit Lua\LaTeX und erwartet eine hinreichend
moderne \LaTeX-Version, mindestens vom Juli 2022.
Es lädt folgende Pakete:
\begin{itemize}
  \item \pkg{etoolbox} 
  \item \pkg{luacode}
  \item \pkg{tikz}
  \item \pkg{translator}
  \item \pkg{xstring}
\end{itemize}
Diese Pakete sind in allen gängigen \TeX-Distributionen vorhanden.

\section{Nutzung}
Das Paket wird auf dem üblichen Weg mit

\centerline{\cs{usepackage}\oarg{optionen}\{\thepkg\}}

geladen.
Wenn der genutzte Raum ist bereits in der Datenbank von \thepkg\ vorhanden ist,
kann man ihn mit der Option

\Option{room=}\WithValues{\meta{Raumnummer}}
festlegen.
Ansonsten


\end{document}
