\documentclass{scrartcl}
% Wir versuchen das Papier bestmöglich auszunutzen.
\usepackage[a4paper,landscape,inner=10pt,outer=10pt,top=1cm,bottom=1cm]{geometry}

% Der Raum ist noch nicht aufgenommen, wir müssen ihn selbst "bauen".
\usepackage[
shape=arc,  % Sitzreihen bilden ein Bogen
rows=12,          % aus 10 Reihen
seats per row=34, %
init,
blackboard,       % Wir wollen eine Tafel
rownumbers = both,
]{tucseating}

\begin{document}
\centering\textbf{\Huge\sffamily Sitzplan}\bigskip

\tucsInitSeating
\removeSeats{{1,1},{1,2},{1,3},{1,4},{1,5},{1,6},{1,-1},{1,-2},{1,-3},{1,-4},{1,-5},{1,-6}}
\removeSeats{{2,1},{2,2},{2,3},{2,4},{2,5},{2,-1},{2,-2},{2,-3},{2,-4},{2,-5}}
\removeSeats{{3,1},{3,2},{3,3},{3,4},{3,-1},{3,-2},{3,-3},{3,-4}}
\removeSeats{{4,1},{4,2},{4,3},{4,4},{4,-1},{4,-2},{4,-3},{4,-4}}
\removeSeats{{5,1},{5,2},{5,3},{5,-1},{5,-2},{5,-3}}
\removeSeats{{6,1},{6,2},{6,-1},{6,-2}}
\removeSeats{{7,1},{7,-1}}
\removeSeats{{8,1},{8,-1}}
\removeSeats{{9,1},{9,-1}}
\removeSeats{{11,15},{11,16},{11,17},{11,18},{11,19},{11,20}}
\removeSeats{{12,15},{12,16},{12,17},{12,18},{12,19},{12,20}}

%\removeSeatAt{1}{-1}
% Damit ist das Raumlayout fertig.
% Wir wollen aber noch ein paar Anpassungen bei der Darstellung vornehmen:
\tucsConfig{
  seat label font=\sffamily\bfseries,
  seat label color=blue
}


% Nun können wir uns an das Sitzschema machen
\setupSeatingScheme{
  aisle restarts scheme,
  row sep=1,
  %row restart after=4,
  start row=1,
  ignore removed seats
}
\seatingSchemeInRows{X--}
%\scalebox{.7}{
  \tucsDrawSeating
%}
\end{document}

