\documentclass{scrartcl}
% Wir versuchen das Papier bestmöglich auszunutzen.
\usepackage[a4paper,landscape,inner=10pt,outer=10pt,top=1cm,bottom=1cm]{geometry}

% Der Raum ist noch nicht aufgenommen, wir müssen ihn selbst "bauen".
\usepackage[
shape=rectangle,  % Sitzreihen bilden ein Rechtecke
rows=10,          % aus 10 Reihen
seats per row=14, % mit 12 sitzen und 2 Zwischengängen
blackboard,       % Wir wollen eine Tafel
rownumbers = both,
]{tucseating}

\begin{document}
\centering\textbf{\Huge\sffamily Sitzplan}\bigskip

\tucsInitSeating
% Zwei Mittelgänge
\setAisle{5}
\setAisle{10}
% Die erste Reihe ist kürzer
\removeSeatAt{1}{1}
\removeSeatAt{1}{-1}
% Damit ist das Raumlayout fertig.
% Wir wollen aber noch ein paar Anpassungen bei der Darstellung vornehmen:
\tucsConfig{
  seat label font=\sffamily\bfseries,
  seat label color=blue
}


% Nun können wir uns an das Sitzschema machen
\setupSeatingScheme{
  aisle restarts scheme,
  row restart after=4,
  start row=2
}
\seatingSchemeInRows{X--}
\tucsDrawSeating

\end{document}

